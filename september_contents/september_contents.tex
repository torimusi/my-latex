\documentclass[a4paper, uplatex]{jsarticle}
\usepackage{style/mylayout}

%--------------------------------------------------
\begin{document}

%--------------------------------------------------
% 日付{年}{月}{日} ※要変更
\symposiumdate{20??}{09}{??}

%--------------------------------------------------
% 発表会タイトル{研究室合同中間発表会}{20??年度 プログラム} ※要変更
\vspace{1cm}
\symposiumtitle{研究室合同中間発表会}{20??年度 プログラム}

%--------------------------------------------------
% ロゴの挿入
\vspace{3cm}
\begin{figure}[h]
  \centering
  \divide\textwidth by 2
  \includegraphics[width=\textwidth]{img/logo_doshisha.pdf}
\end{figure}

%--------------------------------------------------
% 出版元(研究室名)
\vspace{2cm}
\publisher

%--------------------------------------------------
% 改ページ
\newpage

%--------------------------------------------------
% 空白ページの挿入 ※必要なら変更
% 各論文が、PDFは奇数ページから、紙媒体は偶数ページから始まるよう適宜調整してください
 \newpage % 全角スペースは消去せず "%" だけ追加・消去してください

%--------------------------------------------------
% 目次見出し
\makeheading

%--------------------------------------------------
% 項目一覧 ※要変更
% \session{セッション名}
% \content{発表時間}{論文タイトル}{著者}{ページ番号}

\vspace{0.5cm}

\session{SYS1}
\begin{description}

  \setlength{\leftskip}{-1.7zw}

  \content{9:30〜9:45}{タイトルタイトルタイトルタイトルタイトルタイトルタイトルタイトル}{田中 太郎}{1}
  \content{10:15〜10:30}{タイトルタイトルタイトルタイトルタイトルタイトルタイトルタイトルタイトルタイトル}{山田 花子}{3}

\end{description}

\session{UI}
\begin{description}

    \setlength{\leftskip}{-1.7zw}
  
    \content{10:45〜11:00}{タイトルタイトルタイトルタイトルタイトルタイトル}{鈴木 一郎}{5}
    \content{11:45〜12:00}{タイトルタイトルタイトルタイトルタイトルタイトルタイトルタイトルタイトルタイトルタイトル}{伊藤 愛}{7}
  
  \end{description}

\session{SYS2}
\begin{description}
  
      \setlength{\leftskip}{-1.7zw}
    
      \content{13:15〜13:30}{タイトルタイトルタイトルタイトルタイトル}{高橋 裕太}{9}
      \content{14:15〜14:30}{タイトルタイトルタイトル}{齊藤 香織}{11}
    
\end{description}

\session{COMM}
\begin{description}

    \setlength{\leftskip}{-1.7zw}
  
    \content{15:00〜15:15}{タイトルタイトルタイトルタイトルタイトルタイトルタイトルタイトルタイトルタイトルタイトルタイトルタイトル}{小林 健一}{13}
    \content{15:30〜15:45}{タイトルタイトルタイトルタイトルタイトル}{馬場 優香}{15}
  
\end{description}

%--------------------------------------------------
\end{document}

%--------------------------------------------------