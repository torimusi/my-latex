\documentclass[a4paper,10pt,uplatex]{jsarticle}
\usepackage{url}

\title{通信工学特論 期末レポート}
\author{情報工学専攻 1316220136 国本 典晟}
\date{}
\begin{document}
\maketitle

\section{6Gの展望}
\subsection{6Gの動機}
第5世代移動通信システム(5G)が商用化されて数年経つが,一般人の実感できる変化はスマートフォンが5Gに対応したことくらいであり,5G商用化前に実用化できるとされていたことはまだ発展途上である.
中でも,自動運転の推進やスポーツ観戦の多角化などは,低遅延かつ高信頼な通信が必要であり,現在の5Gでも両立して実現するのは難しい.
今後新たなビジネスを創出し社会のデジタルトランスフォーメーションを推進するためには,5Gの高度化および6Gが求められると言われている\cite{ドコモ,KDDI}.
6G自体はあくまで通信技術であるが,6Gにより大容量データを高速かつ安定して通信することで,様々なサービスの要件を満たすことができる.
本稿では,6Gにより実現および発展が期待されるWeb3.0およびメタバースについて考えを述べる.

\subsection{Web3.0}
Web3.0とは,ブロックチェーンプラットフォーム「Ethereum」共同経営者のGavin James Woodが2014年に発表した造語であり,「ブロックチェーンに基づく分散型オンライン・エコシステム」を指す\cite{Web3.0}.
Web2.0では,SNSなどの普及により双方向のコミュケーションが可能になったが,巨大なプラットフォーマーに個人データが集中する仕組みが作られたことで,データの独占による透明性の欠如やコミュニティとの利害相反が問題となった.
それに対しWeb3.0は,ブロックチェーンを用いて個人がデータを所有・管理し,中央集権不在で個人同士が自由に交流・取引を行うことで,新たな経済活動のフロンティアとなることが期待されている\cite{経産省}.
しかし,ブロックチェーンでは取引の記録をブロックと呼ばれる塊に格納して,ブロックを時系列に沿って繋げていくため,ネットワークの参加者が増えることでトランザクションの量が増え,トランザクションの処理速度が十分に担保されなくなる危険がある.
そのため,6Gによりネットワークに参加するノードを効率的に管理し,より多くのデータをリアルタイムで収集することで,ブロックチェーンの拡張性を向上することが期待されている.\par
また,6Gの通信には,高い通信品質の提供やプライバシ保護のために計算資源の効率的な管理と共有が問題となるが,その解決策としてブロックチェーンの活用が研究されている\cite{資源管理,長谷川}.
5Gとブロックチェーンが相互補完できる関係にあった\cite{ASCII}ように,5Gより低遅延・高信頼の通信を可能にする6Gにおいてもブロックチェーンと相互補完できる関係にあると言える.

\subsection{メタバース}
メタバースとは,SF作家のNeal Stephensonが1992年に発表した小説『Snow Crash』に登場する架空の仮想空間サービスの名称であり,その後仮想空間サービスの総称や仮想空間自体の名称として用いられるようになった.
Facebookが2021年10月に社名を「Meta」に変更するなど,昨今話題の言葉であるが決して最新技術というわけではない.
1990年代後半頃からMMORPGと呼ばれるユーザがアバターを介してコミュニケーションを行うゲームをベースに普及し,2006年にブームとなった3D仮想空間サービス「Second Life」により世界的に注目されるようになった.
当時はまだスマートフォンもあまり普及していない時代であったが,昨今のメタバースの注目の背景にはVR/AR技術の発達がある.
VR/AR技術を活用したメタバースが実現した世界では,スマートグラスと呼ばれる眼鏡型デバイスを装着することで,現実世界と仮想世界を融合しリアルタイムでインタラクションを行うことができる.
しかし,こうした世界を実現するためには,高い柔軟性と堅牢性を兼ね備えたネットワークが必要となる.
5Gの高度化および6Gにより上り帯域幅の拡張やさらなる低遅延を実現することで,大規模なメタバースの構想が推進すると期待されている\cite{Nokia}.

\subsection{6Gにより実現され得る世界の考察}
\label{6Gにより実現され得る世界の考察}
6Gの普及によりWeb3.0およびメタバースの技術が発展することで実現され得る世界について考察する.
現代では誰もが一人一台スマートフォンを持つ時代であるが,一人一台持つデバイスがスマートグラスに置き換わる.
スマートグラスには現実世界と仮想空間を融合するために必要な最低限の入出力機能と通信機能を持つのみであり,余計なCPUやメモリを持たない.
現実世界には計算資源やメモリとなるインフラとしてエッジコンピュータが至る所に設置され,スマートグラスはエッジコンピュータと通信することでVR体験やAR表示を可能にする.
メタバースは新たなSNSの形となり,アイコンやアカウント名ではなく各々の3Dアバターを介してコミュニケーションが行われる.
現在のテレワークはメールやチャットなどのテキストコミュニケーションが主流であるために仕事効率の低下が問題となっている\cite{Microsoft}が,アバターを介したコミュニケーションにより,実際に会って会話をするような効果が期待できる.
Web3.0により個人間の交流が活発になり,個人によるサービスの創出やそれに対する代金の支払いはプラットフォームを介することがなくなる.
スマートグラスによる現実世界と仮想空間の融合はエンターテイメント業界にも活用され,家にいながらまるで目の前でスポーツの試合やアーティストのライブが行われているかのような没入感を楽しむことができる.\par
こうした世界の実現には,スマートグラスへの入力をエッジコンピュータへ送信し,エッジコンピュータにて計算処理を行い,その結果をスマートグラスへ送信する一連の処理を極めて高速に行う必要がある.
また,個人間の金銭的取引をプラットフォームなく行うため,その通信の信頼性やセキュリティの堅牢性は高いものでなくてはならない.
6Gの実現により,こうした要件を満たすことができると考えられる.

\section{6Gの懸念点}
6Gが実現されれば\ref{6Gにより実現され得る世界の考察}節で述べたような世界の実現が予想されるが、それに伴う格差の拡大が懸念される.
スマートグラスのようなデバイスにより人間の知覚の範囲を延長するのは,健康な身体や精神の機能を向上させるエンハンスメントの一つであると言える.
エンハンスメントは最新技術を利用したものであることから,本来援助を受けるべきである金銭的に貧しい人々や身体的・精神的に障がいを持つ人々はその恩恵を享受しづらく,先進国の富を持った人々が真っ先にその恩恵を受けられる.
そのため,今ある格差をさらに拡大するものであるという懸念があり,その正当性について議論が行われている\cite{エンハンスメント}.
\ref{6Gにより実現され得る世界の考察}節で述べたような世界も同様に,エッジコンピュータや通信設備などのインフラ整備に莫大な資金が必要となるため,まず先進国において実現されることが予想される.
先進国において現実世界と仮想世界の融合が実現されれば,先進国での教育や医療水準,仕事やコミュニケーションの形が発展途上国と全く異なる次元に変わることになり,その格差の是正は絶望的になる事態が懸念される.\par
一方で,6Gによる格差の是正の可能性もあると考えられる.
\ref{6Gにより実現され得る世界の考察}節で述べたスマートグラスは,CPUやメモリといった機能をエッジコンピューティングを持たないことから,デバイス自体の価格はスマートフォンやPCと比較して大きく下がることが予想され,発展途上国の人々にも先進国からの援助により一人一台配布できる可能性がある.
また,発展途上国では地理的要因から従来の通信設備を整えることが難しい場合が多いが,6Gにおいて実現が期待されている「空飛ぶ基地局」を用いることで,地上の制約を受けることなく通信インフラを用意することが期待されている\cite{空飛ぶ基地局}.
6Gを適切に活用することで,世界の情報技術の格差の是正が期待される.

\footnotesize{
  \begin{thebibliography}{99}
    \bibitem{ドコモ} 株式会社NTTドコモ,"ホワイトペーパー:5Gの高度化と6G"(4.0版),\url{https://www.docomo.ne.jp/corporate/technology/whitepaper_6g/},Nov. 2021.
    \bibitem{KDDI} KDDI株式会社,"ホワイトペーパー:Beyond 5G/6G"(2.0.1版),\url{https://www.kddi-research.jp/tech/whitepaper_b5g_6g},Oct. 2021.
    \bibitem{Web3.0} " Gilad Edelman, "The Father of Web3 Wants You to Trust Less," WIRED, \url{https://www.wired.com/story/web3-gavin-wood-interview/}, Nov. 2021.
    \bibitem{経産省} 経済産業省,"経済秩序の激動期における経済産業政策の方向性," May. 2022.
    \bibitem{資源管理} H. Xu, P.V. Klaine, O. Onireti, B. Cao, M. Imran and L. Zhang, "Blockchain-enabled resource management and sharing for 6G communications," Digital Communications and Networks, vol.6, no.3, Aug. 2020.
    \bibitem{長谷川} " 大阪大学大学院情報科学研究科情報流通プラットフォーム講座長谷川研究室,"6G × プライバシー," \url{https://www-hasegawa.ist.osaka-u.ac.jp/research-privacy.html},Jul. 2022.
    \bibitem{ASCII} " 久我 吉史,"次世代通信規格5Gがブロックチェーンに与える恩恵と、5G自体への好影響," ASCII FinTech,\url{https://ascii.jp/elem/000/001/987/1987000/},Dec. 2019.
    \bibitem{Nokia} " 加藤 樹子,"通信大手ノキアのメタバース構想、6G活用で2030年から本格化," 日経XTECH,\url{https://xtech.nikkei.com/atcl/nxt/news/18/12879/},May. 2022.
    \bibitem{Microsoft} L. Yang, D. Holtz, S. Jaffe, S. Suri, S. Sinha, J. Weston, C. Joyce, N. Shah, K. Sherman, B. Hecht and J. Teevan, "The effects of remote work on collaboration among information workers," Nature Human Behaviour, vol.6, pp.43-54, Jan. 2022.
    \bibitem{エンハンスメント} 堀内 進之介, "道徳的エンハンスメントによる共助的な社会関係の底上げの可能性------テクノ進歩派の理論的根拠に関する検討," Japanese Journal for Science, Technology \& Society, vol.29,pp.31-49,2020.
    \bibitem{空飛ぶ基地局} " NHK,"「空飛ぶ基地局」開発本格化 災害時や途上国での活用に期待," \url{https://www3.nhk.or.jp/news/html/20220124/k10013446171000.html},Jan. 2022.
  \end{thebibliography}
}

\end{document}