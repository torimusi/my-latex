\documentclass[a4paper, uplatex]{jsarticle}
\usepackage{style/mylayout}
\usepackage{layout}

%--------------------------------------------------
\begin{document}
%--------------------------------------------------
% 見出し{発表会名} ※要変更
\makeheading{第??回月例発表会}
\vspace{0.5cm}

%--------------------------------------------------
% 項目一覧{タイトル}{著者}{ページ番号} ※要変更
\begin{description}

  \setlength{\leftskip}{-3zw}

  \content{うまく研究テーマを捻り出す手法}{田中 太郎}{1}
  \content{中身のないテーマで月例発表会を乗り切る手法}{山田 花子}{3}
  \content{進捗報告までの本当の進捗の評価}{鈴木 一郎}{5}
  \content{研究の重要性と実現性の評価と検討}{伊藤 愛}{7}
  \content{ネタ切れを考慮した研究テーマ名の提案}{高橋 裕太}{9}

\end{description}

\layout

%--------------------------------------------------
\end{document}
%--------------------------------------------------


