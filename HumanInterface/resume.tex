\documentclass[a4paper,10pt,twocolumn,uplatex]{jsarticle}
\usepackage{style/nislab,style/resume}

%---------------------------------------------------------------------
% レジュメ種別・日付設定(要変更)
% \type{} 1:修士論文諮問会 2:卒業論文発表会 3:月例発表会 4:研究室合同発表会
%---------------------------------------------------------------------
\type{3}
\year{2023}
\month{1}
\date{10}

%---------------------------------------------------------------------
% ページ番号設定(要変更)
%---------------------------------------------------------------------
\setcounter{page}{1}

%---------------------------------------------------------------------
% 変更不要
%---------------------------------------------------------------------
\begin{document}

%---------------------------------------------------------------------
% タイトル作成部分(要変更)
% \maketitle{タイトル}{title}{名前}{name}
%---------------------------------------------------------------------
\maketitle{Multi-Touch Querying on Data Physicalizations in Immersive AR}
{土居 大輝}
{国本 典晟}
{鈴木 彩門}

%---------------------------------------------------------------------
\section{はじめに}
協調型自動運転の実現に向けて,車両とあらゆるものを接続するV2X(Vehicle to Everything)通信の研究が積極的に行われている.
安全かつ効率的な走行のため,車両は基地局を介してインターネット上のクラウドサーバや基地局近くなどの地理的な分散配置が考えられているエッジサーバ\cite{ETSI}と通信を行い,必要な情報を得る.
しかし,今後交通サービスを提供するアプリケーションの増加や自動運転車両の普及が進む中で,ある基地局を介してエッジサーバと通信する車両が,そこで利用可能な通信帯域で収容可能な車両台数以上に集中した場合に,通信のQoS(Quality of Service)を保証することができない事態が発生することが懸念されている\cite{SDNMEC}.
事前にQoSが保証されないことを予測できなかった場合,協調型自動運転に必要な情報を受け取ることができないため,安全性の低下や旅行時間の増加といった問題が発生する.
また,他の車両よりも優先的に情報を受け取る必要がある緊急車両などのQoSを保証できない問題も想定される.\par
本研究では,ソフトウェアを介してネットワークを一元管理するSDN(Software Defined Networking)を利用することで,車両の移動を想定することでQoSを予測することを提案する.
QoSが保証できない場合には,QoS予測に基づき基地局のネットワークを制御する.
また,特定車両に対してQoSの予約を行うことを提案する.

%---------------------------------------------------------------------
\section{提案手法}
\subsection{SDNを利用したQoS予測}
\label{QoSprediction}
SDNコントローラを\figref{fig:architecture}のように配置し,SDNを利用して車両の移動を想定する.
車両は現在地,速度,走行方向,移動計画を常にSDNコントローラへ送信する.
この送信に必要な通信帯域はQoS予測に必要最低限の通信として,全車両分常に確保されているものと考える.
SDNコントローラは自身が管理する範囲内の地図情報と基地局の位置と利用可能な通信帯域の情報を持つ.
SDNコントローラは車両の情報と地図情報を照らし合わせて車両の集中状況を予測する.
予測した集中状況と基地局の情報に基づき,基地局の利用可能な通信帯域で収容可能な台数以上の車両が集中していないかを判定することで,QoSを予測する.

\begin{figure}[t]
	\begin{centering}
    \includegraphics[width=0.8\linewidth]{img/202210_構成図.pdf}
    \caption{提案手法の構成図}
    \label{fig:architecture}
    \end{centering}
\end{figure}

%---------------------------------------------------------------------
\subsection{QoS予測に基づくネットワーク制御}
\label{priority}
\ref{QoSprediction}節のQoS予測の結果,QoSの保証ができない場合,協調型自動運転に必要な情報を受け取ることができないため,自車両に搭載されたセンサのみを用いて慎重に走行する必要があり,旅行時間の増大が懸念される.
そこで,SDNコントローラはネットワーク制御を行うことでQoSの保証を目指す.
ある基地局において,利用可能な通信帯域で収容可能な台数以上に車両が集中している場合,収容可能な台数から超過している台数分の車両を,まだ利用可能な通信帯域に空きのある基地局の周辺を走行するよう経路を変更させることで,QoSを保証することができる.
しかし,経路を変更したために旅行時間が大きく増大し,変更前の経路を自律走行した方が結果として旅行時間が短かった,という可能性がある.
そのため,経路ごとに旅行時間と通信する基地局の空き帯域をパラメータとしてコストを計算し,そのコストをもとに各車両の経路を決定する.\par
QoS予測後のネットワーク制御の手順を\figref{fig:control}に示す.
まず,SDNコントローラはQoS予測の結果からQoSを保証できない車両に対して,上述の経路のコストを計算し,全車両の合計コストが最も小さい組み合わせを経路として決定する.
決定した経路をもとに,SDNコントローラは管理する範囲内の各基地局に対して,各基地局が通信する車両とその時刻の情報を送信する.
また,SDNコントローラは車両に対して決定した経路を送信する.
経路を計算した結果,どの基地局でもQoSを保証できない場合や,自律走行の方が旅行時間が短いと判断した場合には,車両に元の経路を自律走行するように指示する.
その後実際の走行において,SDNコントローラの指示通りに車両と基地局の通信あるいは自律走行を行う.
なお,このネットワーク制御にて行う通信は車両の走行のために最低限必要な情報を扱うため,\ref{QoSprediction}節で述べた車両情報の送信同様,全車両分の通信帯域が事前に確保されているとする.

\begin{figure}[t]
	\begin{centering}
    \includegraphics[width=0.96\linewidth]{img/202210_ネットワーク制御.pdf}
    \caption{ネットワーク制御の手順}
    \label{fig:control}
    \end{centering}
\end{figure}

%---------------------------------------------------------------------
\subsection{QoS予約}
緊急車両などの特定車両は,より安全で効率的な運転のため,他の車両よりも優先して情報を受け取る必要があるため,QoSを事前に保証しなければならない場合が想定される.
そこで,特定車両に対してQoSの予約を行うことを提案する.
QoS予約の手順は次の通りである.
まず,特定車両はSDNコントローラに対して,通信したい基地局・時刻を指定してQoS予約をリクエストする.
SDNコントローラはリクエストされた基地局に対して,QoS予約を行う車両と時刻を送信し,その基地局の利用可能な通信帯域をQoS予約分差し引いて更新する.
最後にSDNコントローラは特定車両に対してQoS予約の完了を送信する.
リクエストされた基地局に既に十分な空き帯域がなくQoS予約ができない場合には,空き帯域がある基地局の情報を車両へ送信し,車両はその情報をもとに再度QoS予約のリクエストを行うか判断する.
なお,QoS予約は\ref{QoSprediction}節のQoS予測の前に同じタイミングで行われることとし,より優先度が高いQoS予約のリクエストが来た場合の完了済みのQoS予約の更新や,QoS予測完了後のQoS予約のリクエストの割り込みは考えない.

%---------------------------------------------------------------------
\section{シミュレーションによる評価}
提案手法の有効性を検証するため,\figref{fig:model}に示すような移動経路網モデルでシミュレーションを行う.
移動経路網モデルはWayPointが格子状に配置され,車両は事前に設定したスタート地点からゴール地点に向けて,WayPointとそれらを結ぶリンク上を移動する.
車両は全て時刻ごとに一定速度で移動し,衝突や渋滞は考えない.
移動経路網モデル上の基地局および車両は1つのSDNコントローラが管理を行うものとし,経路はスタート地点からゴール地点までのものを計算する.
各基地局に収容可能な台数を設定し,基地局周辺の車両台数がそれを超過する場合,超過する分の車両は自律走行を想定して速度を落として移動する.\par
時刻ごとの基地局周囲の車両台数を,提案手法を用いない場合と用いた場合を比較することにより,QoS保証について評価する.
基地局の利用可能な通信帯域で収容可能な台数以上の車両が基地局周囲のいる場合,QoSを保証できていないと考える.
また,経路ごとの旅行時間と通信する基地局の空き帯域をパラメータとしてコストを計算し経路を決定する手法について,提案手法を用いない場合と全車両の合計旅行時間を比較することにより評価を行う.

\begin{figure}[t]
	\begin{centering}
    \includegraphics[width=0.84\linewidth]{img/202210_シミュレーションモデル.pdf}
    \caption{シミュレーションモデル}
    \label{fig:model}
    \end{centering}
\end{figure}

%---------------------------------------------------------------------
\section{まとめと今後の課題}
V2X通信において,基地局周辺に車両が集中した場合,基地局の利用可能な通信帯域で収容可能な台数を超過し,QoSを保証できない問題が懸念されている.
この問題に対して,SDNを用いて車両の集中状況を予測することでQoSを予測し,QoS予測に基づき基地局のネットワークを制御することと,特定車両のQoSを予約することを提案した.
本研究では,車両の経路の変更や自律走行の指示,QoS予約により車両と基地局のネットワークの制御を検討したが,より安全で効率的な走行を行うために,基地局と通信できない車両に対して,車車間通信によって情報提供を行うことも併せて検討する必要があると考えている.

%---------------------------------------------------------------------
% Bibliography(参考文献)
%---------------------------------------------------------------------
% thebibliography を利用する場合は以下を使用
\footnotesize{
  \begin{thebibliography}{99}
    \bibitem{ETSI} ETSI, Multi-access Edge Computing (MEC); Study on MEC Support for V2X Use Cases, 2018.
    \bibitem{SDNMEC} Kousaridas A., Schimpe A., Euler S., et al, 5G cross-border operation for connected and automated mobility: Challenges and solutions, MDPI Future Internet, vol.12, no.1, 2020.
  \end{thebibliography}
}

% BibTex を利用する場合は以下を使用(初めての人には難しいかも)
% \bibliographystyle{junsrt}
% \bibliography{myref}

%---------------------------------------------------------------------
\end{document}
%---------------------------------------------------------------------
